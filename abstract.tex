"A sensation of heat and fullness is experienced in the eyes, first along the edges of the lids, and especially in the inner angles, but after some time over the whole of the ball." is how Dr John Bostock first described what he named "hay fever" in 1819. Hay fever, or seasonal allergic rhinitis as it's referred to in the medical and scientific communities is usually felt as being very similar to the common cold. Symptoms usually start in the eyes, but they can also affect the sinuses, nose, chest and general well-being. Severe cases can leave you feeling extremely fatigued due to bad sleep quality caused by waking up with asthmatic symptoms.\\

We still are not completely sure what causes allergies, it seems that there are so many variables that it is very difficult to make strong conclusions. The aim of my project is to try to visualise allergy symptoms in the UK so that we can try to identify hotspots. Once these hotspots have been identified, we can look into what exists in that area that could be contributing, such as traffic volume, agricultural activity etc.\\

As an allergy sufferer myself who struggles mostly with nasal and chest symptoms, I named my project "Inhale". Inhale has been designed for use by those with a medical or research interest in allergies, but it is also designed to be usable by fellow allergy sufferers.