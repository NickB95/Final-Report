\chapter{Introduction}
\label{cha:intro}

I will briefly explain my motivation for this project and also provide an overview of what you will read in this report.

\section{Motivation} 
I have a long history with allergies. As far back as I can remember I have suffered from allergy symptoms ranging from waking up with gummy eyes to having a permanently sealed nose and tight chest. I've had little luck with medication so I usually end up trying to change my environment to limit the number of allergens with an air filter. Even an industrial air filter does not provide much relief for me.\\

When I saw this project proposed by Dr Markel Vigo I knew it was for me. The challenge of trying to find potential causes for specific allergies was quite exciting. Not only would providing a means for identifying hotspots and their causes help the scientific community, it would also help me personally.\\

The majority of my previous programming experience has been with java within the school and C# when working for companies outside of University. I quickly realised this project would have to be done using some kind of web technology, so that fact I had very little experience with this side of computer science, I decided it'd be a great opportunity to develop my skill sets to help my career.\\

\section{Aim}
\label{sec:aim}

I gave myself main goals that I wanted Inhale to fulfil

\begin{enumerate}
  \item Identify hotspots
  \item Be user friendly
  \item Be useful for future users. Both general public and researchers.
  \item Be accessible
\end{enumerate}


Most emphasis was on the identification of allergy hotspots. I wanted a hotspot algorithm that was both valid and  powerful to be combined with a useful visualisation technique.

%% Before anything data related, explain hotspots

\section{Driving Latex}

\LaTeX\ is not a WYSIWYG system. You first prepare source files,
\textsf{report.tex} etc., similar to the ones here, using your
favourite text editor.

The UNIX commands you need to drive \LaTeX\ are:
\begin{enumerate}
\item \texttt{latex.} Output from \texttt{latex} can be previewed on
  the screen with \texttt{xdvi} or \texttt{gsprev} (under X), and printed on the laserprinter using  \texttt{dvips}.
  \texttt{latex} produces \textsf{.dvi} files which are used by
  \texttt{xdvi} or \texttt{gsprev} and \texttt{dvipr}.  To get all the
  cross references and the table of contents correct you sometimes
  need to run this command twice in succession.  Keep on re-running
  until the advice to rerun at the end of the output goes away.

  Many people now use \texttt{pdflatex} instead of \texttt{latex} to produce \texttt{pdf} files directly.
  
\item \texttt{xdvi} or a pdf viewer such as \texttt{xpdf}, \texttt{evince} or \texttt{Preview} (Mac OS X). Previews the document on the
  screen, again the parameter is \textsf{report} (or \texttt{report.pdf}).
  
\item \texttt{dvips.} This is used to print dvi files on the laserprinter.  The
  parameter is again \texttt{report}. Most pdf previewers provide their own printing interface.
  
\item \texttt{xfig}. Can be used to produce diagrams, see
  section~\ref{sec:diagrams}.

\end{enumerate}
There are on-line manual pages for each of the commands described above.

The set of files used to produce this document is in
\textsf{/opt/info/doc/latex} (in one of the subdirectories
\textsf{3rd-yr} or \textsf{MU-Thesis}).  They are also on the web at \url{http://studentnet.cs.manchester.ac.uk/resources/latex/MUThesis/}. 

You will find that for more detailed points you will need to refer to
Lamport's `\LaTeX\ a document preparation system'~\cite{lamport}
(Copies in the library  in Blackwell's).
This example has been written using the most recent version of \LaTeX
(LaTeX2e), which is described in the \emph{2nd edition} of Lamport's
book. Buying a cheap copy of the 1st edition is probably not a good
investment. An alternative to Lamport's book, which some people
prefer, is `A guide to {\LaTeXe}' by Kopka and Daly~\cite{kopka}. The
older `A guide to {\LaTeX}'~\cite{kopka-old} by the same authors is
now obsolete. There is also plenty of support material for \LaTeX\ on the web.

\section{Software Environment}
\subsection{Occam}

Here is some more example text, showing various \LaTeX\ facilities
you may need.  The project was mostly programmed in
\textbf{Occam}~\cite{occam}.  (A citation has been created  here to an
entry in the bibliography at the end of the report. See
chapter~\ref{cha:bib} for more details on how to do this).

Note the way of getting boldface, \textit{textit} is used for italic.
\emph{emph} is used for emphasis and is the same as italic except when
already in italic.  Note the cross reference to the bibliography.
This is how you create a footnote: DMA\footnote{Direct Memory Access.
  Footnotes can stretch over more than one line if you have a lot to
  say, but be careful not to overdo them.}.

Here is a reference to a figure. See figure~\ref{pipeline}.
\begin{figure}[htbp]
  \centering
  \setlength{\unitlength}{0.0125in}
\begin{picture}(300,35)(60,730)
\thicklines
\put(340,750){\vector( 1, 0){ 20}}
\put( 80,740){\framebox(20,20){}}
\put( 60,750){\vector( 1, 0){ 20}}
\put(100,750){\vector( 1, 0){ 20}}
\put(120,740){\framebox(20,20){}}
\put(180,750){\vector( 1, 0){ 20}}
\put(200,740){\framebox(20,20){}}
\put(160,740){\framebox(20,20){}}
\put(140,750){\vector( 1, 0){ 20}}
\put(240,740){\framebox(20,20){}}
\put(220,750){\vector( 1, 0){ 20}}
\put(260,750){\vector( 1, 0){ 20}}
\put(280,740){\framebox(20,20){}}
\put(320,740){\framebox(20,20){}}
\put(300,750){\vector( 1, 0){ 20}}
\end{picture}

  \caption{A Pipeline of processors
    \label{pipeline}}           %  this label must appear after the
                                %  \caption, and before the end of the
                                %  figure
\end{figure}
The picture in this figure was created the hard way using the picture
facility of \LaTeX. It is \emph{much} easier to use \texttt{xfig}, as
described in section~\ref{sec:diagrams}.  Whatever its contents, a
figure `floats' to a `suitable' point in the text and is never split
across a page boundary. (\LaTeX's idea of what constitutes a suitable
point may not coincide with yours)


Now we have a verbatim environment; this is a useful way of including
snippits of program, printed in a fixed width font exactly as typed:

\begin{verbatim}
{{{ An example of some folds
...  This is some folded code
  {{{ This is another fold
  This is text within the fold that has now been
  opened so that the text can be read.
  }}}
}}}
\end{verbatim}

% Everything below here is commented material which is used by the
% emacs tex support system called auctex. If you're not an emacs user
% you can safely ignore it. If you do use emacs you should take a look
% at the local emacs or LaTeX WWW pages for more on emacs support for
% LaTeX.

% Local Variables:
% mode: latex
% TeX-master: "report"
% End:

