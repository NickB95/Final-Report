\chapter{Introduction}
\label{cha:intro}

I will briefly explain my motivation for this project, my initial aims also provide an overview of what you will read in this report.

\section{Motivation} 
I have a long history with allergies. As far back as I can remember I have suffered from allergy symptoms ranging from waking up with gummy eyes to having a permanently sealed nose and tight chest. I've had little luck with medication so I usually end up trying to change my environment to limit the number of allergens with an air filter. Even an industrial air filter does not provide much relief for me.\\

When I saw this project proposed by Dr Markel Vigo I knew it was for me. The challenge of trying to find potential causes for specific allergies was quite exciting. Not only would providing a means for identifying hotspots and their causes help the scientific community, it would also help me personally.\\

The majority of my previous programming experience has been with java within the school and C\# when working for companies outside of University. I quickly realised this project would have to be done using some kind of web technology, so the fact I had very little experience with this side of computer science, I decided it'd be a great opportunity to develop my skill sets to help my career.\\

\section{Aim}
\label{sec:aim}

I gave myself the following four main goals that I wanted Inhale to fulfil;

\begin{enumerate}
  \item Visualise hotspots
  \item Be user friendly
  \item Be useful for future users. Both general public and researchers.
  \item Be accessible
\end{enumerate}


The most emphasis was on the identification of allergy hotspots. I wanted a hotspot algorithm that was both valid and  powerful to be combined with a useful visualisation technique. As I want Inhale to be useful for the general public, it's important that the visualisation technique used both looks good and accurately displays the hotspots generated by the algorithm.\\

%% Before anything data related, explain hotspots

\section{Report Overview}

There are no hard to understand, advanced computer science algorithms or techniques used in this project. However, I do make some decisions that will require some background knowledge and context of the related subjects in order to fully understand my reasoning. So the second chapter will provide a brief history of allergies and how certain types of allergies are relevant to my project.\\

I'll talk about the structure and content of the datasets used throughout and cover what similar applications already exist.\\

Once the basics are understood, I will take the reader through the design decisions and implementation stages of Inhale in Chapters 3 and 4. I will go into some detail of the core processes, but nothing too heavy.\\

Chapter 5 will explain how I did my testing and how I interpreted the resulting heatmap displayed by Inhale. I'll show you some of the interesting hotspots and how I evaluated them.\\

I'll conclude the report with a summary of my personal experience developing Inhale and my findings using the final product.\\
    

\section{Software Environment}

My environment options for Inhale were to make a Desktop Application for either Windows or MacOS, website, mobile app or a web application.\\

I decided against making a desktop app for any platform as by going with this route it limits the number of users instantly. Whilst the last two versions of Windows make up around 65\% of the Desktop market, that's still a potential 35\% of users that are completely unable to use my software \cite{windows}. As this project is likely to be shown to potential employers, it would be nice to have it on a platform that is easier to setup and get running.\\

A mobile application was a strong contender, with 

\subsection{Occam}

Here is some more example text, showing various \LaTeX\ facilities
you may need.  The project was mostly programmed in
\textbf{Occam}~\cite{occam}.  (A citation has been created  here to an
entry in the bibliography at the end of the report. See
chapter~\ref{cha:bib} for more details on how to do this).

Note the way of getting boldface, \textit{textit} is used for italic.
\emph{emph} is used for emphasis and is the same as italic except when
already in italic.  Note the cross reference to the bibliography.
This is how you create a footnote: DMA\footnote{Direct Memory Access.
  Footnotes can stretch over more than one line if you have a lot to
  say, but be careful not to overdo them.}.

Here is a reference to a figure. See figure~\ref{pipeline}.
\begin{figure}[htbp]
  \centering
  \setlength{\unitlength}{0.0125in}
\begin{picture}(300,35)(60,730)
\thicklines
\put(340,750){\vector( 1, 0){ 20}}
\put( 80,740){\framebox(20,20){}}
\put( 60,750){\vector( 1, 0){ 20}}
\put(100,750){\vector( 1, 0){ 20}}
\put(120,740){\framebox(20,20){}}
\put(180,750){\vector( 1, 0){ 20}}
\put(200,740){\framebox(20,20){}}
\put(160,740){\framebox(20,20){}}
\put(140,750){\vector( 1, 0){ 20}}
\put(240,740){\framebox(20,20){}}
\put(220,750){\vector( 1, 0){ 20}}
\put(260,750){\vector( 1, 0){ 20}}
\put(280,740){\framebox(20,20){}}
\put(320,740){\framebox(20,20){}}
\put(300,750){\vector( 1, 0){ 20}}
\end{picture}

  \caption{A Pipeline of processors
    \label{pipeline}}           %  this label must appear after the
                                %  \caption, and before the end of the
                                %  figure
\end{figure}
The picture in this figure was created the hard way using the picture
facility of \LaTeX. It is \emph{much} easier to use \texttt{xfig}, as
described in section~\ref{sec:diagrams}.  Whatever its contents, a
figure `floats' to a `suitable' point in the text and is never split
across a page boundary. (\LaTeX's idea of what constitutes a suitable
point may not coincide with yours)


Now we have a verbatim environment; this is a useful way of including
snippits of program, printed in a fixed width font exactly as typed:

\begin{verbatim}
{{{ An example of some folds
...  This is some folded code
  {{{ This is another fold
  This is text within the fold that has now been
  opened so that the text can be read.
  }}}
}}}
\end{verbatim}

% Everything below here is commented material which is used by the
% emacs tex support system called auctex. If you're not an emacs user
% you can safely ignore it. If you do use emacs you should take a look
% at the local emacs or LaTeX WWW pages for more on emacs support for
% LaTeX.

% Local Variables:
% mode: latex
% TeX-master: "report"
% End:

