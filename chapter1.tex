\chapter{Introduction}
\label{cha:intro}

I will briefly explain my motivation for this project, my initial aims and provide an overview of what you will read in the rest of this report.

\section{Motivation} 

I have always suffered with allergy symptoms, ranging from a tight chest, to being completely unable to breathe through my nose for long periods of time. I've had little luck with medication, so I usually try to change my environment to limit the number of allergens, with an air filter for example. However, even an industrial air filter does not provide much relief for me.\\

When this project was proposed by Dr Markel Vigo I knew it was for me. The challenge of trying to find potential causes for specific allergies was quite exciting. Not only would providing a means for identifying hotspots and their causes help the scientific community, it would also help me personally.\\

The majority of my previous programming experience has been with Java within the school and C\# whilst employed. Reading the brief, I knew that I would have to use something I hadn't used before. I thought it would be a great opportunity to explore new technologies and skill sets to help my future career.\\

\section{Aim}
\label{sec:aim}

I gave myself the following four main goals that I wanted Inhale to fulfil, in priority order;

\begin{enumerate}
  \item Accurately visualise hotspots
  \item Be accessible
  \item Be useful for researchers
  \item Be user friendly
\end{enumerate}


The greatest emphasis was on the identification of allergy hotspots. I wanted a hotspot algorithm that was both valid and  powerful to be combined with a useful visualisation technique. I wanted Inhale to be useful for the general public, so it was important that the visualisation techniques used both looks good and accurately displays the hotspots identified by the algorithm.\\

%% Before anything data related, explain hotspots

\section{Report Overview}

There are no hard to understand, advanced computer science algorithms or techniques used in this project. However, I have made some decisions that will require some background knowledge and context of the related subjects to fully understand my reasoning. For this reason, the second chapter will provide a brief history of allergies and how certain types of allergies are relevant to my project.\\

I will discuss the structure and content of the datasets used throughout, and cover what similar applications already exist. Once the basics are covered, I will take the reader through the design decisions and implementation stages of Inhale in Chapters 3 and 4 where I will go into some detail of the core processes.\\

Chapter 5 will explain how I did my testing and how I interpreted the resulting heatmap displayed by Inhale. I will cover some of the most interesting hotspots and how I evaluated them. This report will be concluded with a summary of my personal experience developing Inhale and my findings using the final product.\\