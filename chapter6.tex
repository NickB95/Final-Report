\chapter{Conclusion}
\label{cha:conc}

\section{Conclusion}

Looking back to the goals I set myself when starting this project, I'm pretty happy with what I have achieved. Of course I could always do more, but given the time allocated to this project I managed to satisfy my goals.\\

Inhale takes a geographical allergy symptom data set and successfully displays the results of a hotspot identification algorithm to a user friendly, accessible map available to all. I have prepared and displayed a Road Traffic dataset that can be compared in an attempt to link the symptoms and cause. The datasets provided can be used to highlight anomalous areas for further investigation into what might be causing issues for people in that area.\\

For users who want a bit more from Inhale, they can upload their own datasets to compare to the two main default sets. I will keep developing this project as a side interest, I've recently been in contact with the school of Earth and Environmental Sciences, they will be providing some more datasets for use in future versions.\\

Taking a fairly large project like this from start to finish is always great fun, and a massive learning experience. I've gone from having never used javascript before to using it extensively for the entire project. I've learnt so much about allergies whilst doing my background research, some of which has helped me personally. I found that the prescribed medication I use to help my allergic rhinitis was both less effective, and more expensive than some that is available over the counter.\\

Given more time, I would have liked to get the user interface a bit smoother to allow more intuitive use. Without the walk-through, it can be hard to understand what is happening. I would also like to have more of the algorithm parameters publicly changeable so that research users can further customise their experience. Allowing users to change the hotspot array resolution from the default 500x500 along with the Analysis Field formula on the fly would be a nice addition.\\

